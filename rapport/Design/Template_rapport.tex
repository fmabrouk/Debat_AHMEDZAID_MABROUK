\documentclass[a4paper, 11pt,twoside=true]{scrartcl} % Remplace la classe "article", correspond au standard européen.
\usepackage{Rapport2}
%\usepackage[toc]{multitoc} double table des matières si nécessaire
\usepackage{lipsum}
\clearpairofpagestyles


%%% Paramètres pour les footers et la page de titre
\newcommand{\lab}{ Moteur de résolution de débat }
\newcommand{\tit}{Rapport de projet  phase \no 2}
\newcommand{\nom}{MABROUK Fayez \\ AHMED-ZAID Macyl}
\newcommand{\superv}{Superviseur}
\newcommand{\depa}{Département}

\input{headtitre.tex}



\begin{document}


%%%%TITRE

\maketitle

\tableofcontents % table des matières
\newpage

%%%%%%% Début du document

\section{Introduction}
%\lipsum[7]
Notre projet consiste à chercher des solutions à un certain débat en se basant sur les arguments
de personnes qui se contredisent.
Ce document contiendra 2 parties explicatives :
\begin{itemize}
	\item[* ] La première qui définira les démarches de programmation suivies en représentant
	un diagramme de classe.
	\item[* ] La deuxième listera les différents packages utilisés, les classes, et expliquera la fonction de
	chaque méthode.
\end{itemize}

\section{Démarche de programmation}
\subsection{Diagramme de classe}
\section{Documentation technique}

\section{Conclusion}




%\begin{thebibliography}{widest entry}


%\end{thebibliography}	






\end{document}
